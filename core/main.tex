\chapter{Основная часть}
\section{Пример Кода}
\noindent\indent Для того, что бы первый абзац в блоке был с красной строки необходимо написать \codepart{//noindent//indent}
\section{Секция с подсекциями}
\subsection{Подсекция 1}
\noindent\indent Для того, что бы первый абзац в блоке был с красной строки необходимо написать \codepart{//noindent//indent}
\subsection{Подсекция 2}
\noindent\indent Для того, что бы первый абзац в блоке был с красной строки необходимо написать \codepart{//noindent//indent}

\chapter{Еще одна глава}
\section{Пример Кода}
\noindent\indent Для того, что бы первый абзац в блоке был с красной строки необходимо написать \codepart{//noindent//indent}
\section{Секция с подсекциями}
\subsection{Подсекция 1}
\noindent\indent Для того, что бы первый абзац в блоке был с красной строки необходимо написать \codepart{//noindent//indent}
\subsection{Подсекция 2}
\noindent\indent Для того, что бы первый абзац в блоке был с красной строки необходимо написать \codepart{//noindent//indent}

\chapter*{Глава без номера, но указываемая в оглавлении}
\addcontentsline{toc}{chapter}{Глава без номера, но указываемая в оглавлении}